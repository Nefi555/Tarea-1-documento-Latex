\documentclass[12pt, letter paper]{article}
\usepackage[español]{babel}
\usepackage[T1]{fontenc}
\title{PATRONES}
\date{15/09/2024}
\author{Nefertari Nubia Campos Salazar}
\usepackage{graphicx}
\begin{document}
\maketitle
Los patrones se pueden observar en casi toda expresión matemática, esta nos ayuda  a resolver problemas de aplicación, podemos encontrar patrones tanto en el campo matemático como en el campo visual, a partir de estos podemos determinar un futuro comportamiento de cierta expresión o función matemática que ayuda a deducir ciertas cosas cuando no hay un límite para tal, podemos presentar los números pares por ejemplo:
\begin{center}
2,4,6,8,10,12,14,16,18,20,...2k
\end{center}
Los cuales siguen la la forma 2K donde k es un natural cualquiera.Podemos hacer lo mismo con los impares:
\begin{center}
1,3,5,7,9,11,13,15,17,19,...2k+1
\end{center}
Que siguen la forma de 2k+1 donde k es un némero natural cualquiera y así podemos determinar mas patrones de secuencias numerales como ser multiplos de ciertos números.
\begin{figure}[ht]
    \centering
    \includegraphics[width=0.5\linewidth]{esdlheb_heuristica.jpg}
    \caption{razonamiento lógico ante los patrones matemáticos}
    \label{fig:enter-label}
\end{figure}
\end{document}